% !TEX options=-synctex=-1
\documentclass[a4paper,12pt]{article}
\usepackage[utf8]{inputenc}
\usepackage[T1]{fontenc}
\usepackage[french]{babel}
\usepackage{amsmath,amssymb}
\usepackage{xcolor}
\usepackage{geometry}
\usepackage{soul}
\usepackage{graphicx} % pour inclure des images
\usepackage{animate}

\geometry{
  a4paper,
  total={170mm,257mm},
  left=20mm,
  top=20mm,
}

\usepackage{extensionCours}

\title{TP 4 - Espace}
\author{Groupe 2 \\* Colin, Henri, Leïla, Loïs}

\begin{document}
\maketitle

\sectionExercice{Théorie}
\label{exo:theorique}

Soit un objet défini dans son repère local. On vous fournit les transformations suivantes à appliquer dans l'ordre indiqué :

\begin{itemize}
  \item Une translation par le vecteur \((a, b, c) = (3, -2, 5)\).
  \item Une rotation autour de l'axe \(z\) de \(\theta = \frac{\pi}{3}\) (60°).
\end{itemize}

\begin{enumerate}
  \item Écrivez individuellement la matrice de translation \(T\) et la matrice de rotation \(R_z\) en coordonnées homogènes (matrices \(4 \times 4\)).

    La matrice de translation \(T\) par le vecteur \((a, b, c) = (3, -2, 5)\) en coordonnées homogènes est : 

    \[
    T = 
    \begin{pmatrix}
    1 & 0 & 0 & a \\
    0 & 1 & 0 & b \\
    0 & 0 & 1 & c \\
    0 & 0 & 0 & 1
    \end{pmatrix}
    \]

    Soit :

    \[
    \boxed{
    T = 
    \begin{pmatrix}
    1 & 0 & 0 & 3 \\
    0 & 1 & 0 & -2 \\
    0 & 0 & 1 & 5 \\
    0 & 0 & 0 & 1
    \end{pmatrix}
    }
    \]

    La matrice de rotation \(R_z\) autour de l'axe \(z\) de \(\theta = \frac{\pi}{3}\) en coordonnées homogènes est : 

    \[
    R_z = 
    \begin{pmatrix}
    \cos(\theta) & -\sin(\theta) & 0 & 0 \\
    \sin(\theta) & \cos(\theta) & 0 & 0 \\
    0 & 0 & 1 & 0 \\
    0 & 0 & 0 & 1
    \end{pmatrix}
    \]

    Soit :

    \[
    \boxed{
    R_z = 
    \begin{pmatrix}
    \cos(\frac{\pi}{3}) & -\sin(\frac{\pi}{3}) & 0 & 0 \\
    \sin(\frac{\pi}{3}) & \cos(\frac{\pi}{3}) & 0 & 0 \\
    0 & 0 & 1 & 0 \\
    0 & 0 & 0 & 1
    \end{pmatrix} 
    =
    \begin{pmatrix}
    \frac{1}{2} & -\frac{\sqrt{3}}{2} & 0 & 0 \\
    \frac{\sqrt{3}}{2} & \frac{1}{2} & 0 & 0 \\
    0 & 0 & 1 & 0 \\
    0 & 0 & 0 & 1
    \end{pmatrix} 
    }
    \]

    \newpage

  \item Calculez la matrice de transformation globale \(M\) en composant ces matrices dans l'ordre approprié.

    On obtient par produit $T R_z$ la matrice de transformation globale \(M\) :

    \[
    \boxed{
    M = T R_z =
    \begin{pmatrix}
    \frac{1}{2} & -\frac{\sqrt{3}}{2} & 0 & 3 \\
    \frac{\sqrt{3}}{2} & \frac{1}{2} & 0 & -2 \\
    0 & 0 & 1 & 5 \\
    0 & 0 & 0 & 1
    \end{pmatrix}
    } 
    \]


  \item Appliquez \(M\) à un sommet local 
  \[
  V_{\text{local}} = \begin{pmatrix} 1 \\ 0 \\ 0 \\ 1 \end{pmatrix}
  \]
  pour déterminer sa position dans l'espace monde. \newline

    On obtient les coordonnées mondes par le produit $M V_{\text{local}}$ :

    \[
    V_{\text{monde}} = M V_{\text{local}}
    \]

    \[
    V_{\text{monde}} = 
      \begin{pmatrix}
      \frac{1}{2} & -\frac{\sqrt{3}}{2} & 0 & 3 \\
      \frac{\sqrt{3}}{2} & \frac{1}{2} & 0 & -2 \\
      0 & 0 & 1 & 5 \\
      0 & 0 & 0 & 1
      \end{pmatrix} 
      \times
      \begin{pmatrix} 1 \\ 0 \\ 0 \\ 1 \end{pmatrix}
    \]

    \[
    \boxed{
    V_{\text{monde}} = 
      \begin{pmatrix} \frac{7}{2} \\ -\frac{4 +\sqrt{3}}{2} \\ 5 \\ 1 \end{pmatrix}
    }
    \] \newline

  \item Déterminer le déterminant de la matrice $M$. \newline

    Pour calculer le déterminant d’une matrice \(4 \times 4\), on utilise la méthode de Laplace (développement par cofacteurs), qui consiste à développer le déterminant suivant la première ligne.

    Soit la matrice suivante :

    \[
    \begin{pmatrix}
    a & b & c & d \\
    e & f & g & h \\
    i & j & k & l \\
    m & n & o & p
    \end{pmatrix}
    \]

    Son déterminant se calcule par :

    \[
    \det =
    a \times 
    \begin{pmatrix}
    f & g & h \\
    j & k & l \\
    n & o & p
    \end{pmatrix}
    - b \times 
    \begin{pmatrix}
    e & g & h \\
    i & k & l \\
    m & o & p
    \end{pmatrix}
    + c \times 
    \begin{pmatrix}
    e & f & h \\
    i & j & l \\
    m & n & p
    \end{pmatrix}
    - d \times 
    \begin{pmatrix}
    e & f & g \\
    i & j & k \\
    m & n & o
    \end{pmatrix}
    \]

    Dans notre cas, la matrice \(M\) est :

    \newpage

    \[
    \begin{pmatrix}
    \frac{1}{2} & -\frac{\sqrt{3}}{2} & 0 & 3 \\
    \frac{\sqrt{3}}{2} & \frac{1}{2} & 0 & -2 \\
    0 & 0 & 1 & 5 \\
    0 & 0 & 0 & 1
    \end{pmatrix}
    \]

    On identifie et on applique la formule :

    \[
    \det(M) =
    \frac{1}{2} \times 
    \begin{pmatrix}
    \frac{1}{2} & 0 & -2 \\
    0 & 1 & 5 \\
    0 & 0 & 1
    \end{pmatrix}
    - \left(-\frac{\sqrt{3}}{2}\right) \times 
    \begin{pmatrix}
    \frac{\sqrt{3}}{2} & 0 & -2 \\
    0 & 1 & 5 \\
    0 & 0 & 1
    \end{pmatrix}
    + 0 
    - 3 \times 
    \begin{pmatrix}
    \frac{\sqrt{3}}{2} & \frac{1}{2} & 0 \\
    0 & 0 & 1 \\
    0 & 0 & 0
    \end{pmatrix}
    \]

    On obtient :

    \[
    \det(M) = \frac{1}{2} \times \frac{1}{2} + \frac{\sqrt{3}}{2} \times \frac{\sqrt{3}}{2} = \frac{1}{4} + \frac{3}{4}
    \]

    \[
    \boxed{
    \det(M) = 1
    }
    \] \newline

    \textbf{Conclusion :} le déterminant de la matrice \(M\) est \(1\). La transformation conserve les volumes. \newline

  \item On applique à l'objet \(H(1.5)\) une homothétie uniforme de facteur $1.5$. Déterminer la nouvelle matrice $M$ de transformation globale. \newline

    La matrice d’homothétie uniforme de facteur \(1.5\) en coordonnées homogènes est :

    \[
    H = 
    \begin{pmatrix}
    1.5 & 0 & 0 & 0 \\
    0 & 1.5 & 0 & 0 \\
    0 & 0 & 1.5 & 0 \\
    0 & 0 & 0 & 1
    \end{pmatrix}
    \]

    La nouvelle matrice globale devient donc :

    \[
    M' = T \times R_z \times H
    \]

    \[
    \boxed{
    M' =
    \begin{pmatrix}
    \frac{3}{4} & -\frac{3\sqrt{3}}{4} & 0 & 0 \\
    \frac{3\sqrt{3}}{4} & \frac{3}{4} & 0 & -2 \\
    0 & 0 & \frac{3}{2} & 5 \\
    0 & 0 & 0 & 1
    \end{pmatrix}
    }
    \] \newline

  \item Déterminer le déterminant de la matrice $M$, en déduire une interprétation géométrique du déterminant. \newline

    On applique la même méthode que pour la matrice \(M\), en développant selon la première ligne de :

    \[
    M' = 
    \begin{pmatrix}
    \frac{3}{4} & -\frac{3\sqrt{3}}{4} & 0 & 0 \\
    \frac{3\sqrt{3}}{4} & \frac{3}{4} & 0 & -2 \\
    0 & 0 & \frac{3}{2} & 5 \\
    0 & 0 & 0 & 1
    \end{pmatrix}
    \]

    On effectue les calculs :

    \[
    \det(M') =
    \frac{3}{4} \times 
    \begin{pmatrix}
    \frac{3}{4} & 0 & -2 \\
    0 & \frac{3}{2} & 5 \\
    0 & 0 & 1
    \end{pmatrix}
    + \frac{3\sqrt{3}}{4} \times 
    \begin{pmatrix}
    \frac{3\sqrt{3}}{4} & 0 & -2 \\
    0 & \frac{3}{2} & 5 \\
    0 & 0 & 1
    \end{pmatrix}
    \]

    \[
    = \frac{3}{4} \times \frac{9}{8} + \frac{3\sqrt{3}}{4} \times \frac{9\sqrt{3}}{8}
    \]

    \[
    = \frac{27}{32} + \frac{27}{32} = \frac{54}{32} = \frac{27}{16} 
    \]

    \[
    \boxed{
    \det(M') = \frac{27}{8}
    }
    \] \newline

    \textbf{Conclusion :} le déterminant de la matrice \(M'\) est \(\frac{27}{8} = 3.375\).  
    L’homothétie de facteur \(1{,}5\) multiplie le volume par \(1.5^3 = 3.375\). \newline

  \item Soit les coordonnées monde d'un sommet de l'objet :  
  \[
  W = \begin{pmatrix} 4 \\ -4 \\ 5 \\ 1 \end{pmatrix}
  \]
  déterminer ses coordonnées locales. \newline

    On obtient les coordonnées mondes par le produit $M W$ :

      \[
      V_{\text{local}} = M^{-1} W
      \]

    où $M^{-1}$ est l'inverse de la matrice globale $M$. \newline

    Pour calculer $M^{-1}$, on applique la méthode du pivot de Gauss sur la matrice augmentée $[M \mid I_4]$:

      \[
      \left[
      \begin{array}{cccc|cccc}
      \frac{1}{2} & -\frac{\sqrt{3}}{2} & 0 & 3 & 1 & 0 & 0 & 0 \\
      \frac{\sqrt{3}}{2} & \frac{1}{2} & 0 & -2 & 0 & 1 & 0 & 0 \\
      0 & 0 & 1 & 5 & 0 & 0 & 1 & 0 \\
      0 & 0 & 0 & 1 & 0 & 0 & 0 & 1 \\
      \end{array}
      \right]
      \]

    On obtient :

      \[
      M^{-1} =
      \begin{pmatrix}
      \frac{1}{2} & \frac{\sqrt{3}}{2} & 0 & -\frac{3}{2} + \sqrt{3} \\
      -\frac{\sqrt{3}}{2} & \frac{1}{2} & 0 & \frac{3\sqrt{3}}{2} + 1 \\
      0 & 0 & 1 & -5 \\
      0 & 0 & 0 & 1
      \end{pmatrix}
      \] \newline

    On effectue ensuite le produit matriciel:

      \[
      V_{\text{local}} = 
      \begin{pmatrix}
      \frac{1}{2} & \frac{\sqrt{3}}{2} & 0 & -\frac{3}{2} + \sqrt{3} \\
      -\frac{\sqrt{3}}{2} & \frac{1}{2} & 0 & \frac{3\sqrt{3}}{2} + 1 \\
      0 & 0 & 1 & -5 \\
      0 & 0 & 0 & 1
      \end{pmatrix}
      \begin{pmatrix}
      4 \\ -4 \\ 5 \\ 1
      \end{pmatrix}
      \]

      \[
      \boxed{
      V_{\text{local}} =
      \begin{pmatrix}
      \frac{1}{2} - \sqrt{3} \\
      -1 - \frac{\sqrt{3}}{2} \\
      0 \\
      1
      \end{pmatrix}
      }
      \]

\end{enumerate}

\sectionExercice{Implémentation 3D}
\label{exo:impl}

Dans la suite, on va essayer de reproduire une version simplifié du fonctionnement d'Unity. 

\begin{enumerate}
  \item \emph{Implémenter une classe vecteur et matrice adaptées pour la géométrie dans l'espace et les coordonnées homogènes.}

Pour reprduire uneversion simplifié du moteur Unity, nous avons implémenté une classe \texttt{Vector4} représentant un vecteur à quatre composantes, et sa sous-classe \texttt{HomogeneousVector4} utilisée pour représenter les points en coordonnées homogènes. La classe \texttt{Vector4} fournit une méthode \texttt{multiply\_by\_matrix} permettant de transformer un vecteur par une matrice 4×4.

Nous avons ensuite défini une classe \texttt{Matrix4x4}, représentant une matrice générique en coordonnées homogènes. Elle inclut une méthode \texttt{prod} pour effectuer le produit matriciel ainsi que les diffréntes transformations : \texttt{TranslationMatrix4x4}, \texttt{HomothetyMatrix4x4} et les méthodes de roation selon les 3 axes x, y et z qui sont suffixe du nom de méthode \texttt{RotationMatrix4x4\_}.  

Ces nouvelles classes ont été ajoutées à notre bibliothèque Mathy dans \newline \texttt{geometry\_engine\_librairie}. \newline
  

  \item \emph{Implémenter une façon de calculer les coordonnées barycentrique d'un point pour un triangle donné.}
  

  \item \emph{Implémenter une classe \texttt{gameobject} qui vous servira de réceptacle pour toutes les informations d'un objet dans votre jeu.}
  

  \item \emph{Implémenter une classe \texttt{transform} qui contiendra les informations de géométries de votre \texttt{gameobject}.}
  

  \item \emph{Implémenter une classe \texttt{renderer} qui contiendra les informations de rendu de votre \texttt{gameobject}.}
  

  \item \emph{Implémenter un pipeline de rendu simplifié comme vu en cours pour afficher un cube.}
\end{enumerate}

\end{document}
