% !TEX options=-synctex=-1
\documentclass[a4paper,12pt]{article}
\usepackage[utf8]{inputenc}
\usepackage[T1]{fontenc}
\usepackage[french]{babel}
\usepackage{amsmath,amssymb}
\usepackage{xcolor}
\usepackage{geometry}
\usepackage{soul}

\geometry{
  a4paper,
  total={170mm,257mm},
  left=20mm,
  top=20mm,
}


\usepackage{extensionCours}

\title{TP1 - Algèbre Linéaire et Plan Complexe}
\author{Groupe 2 \\* Colin, Henri, Leïla, Loïs}

\begin{document}
\maketitle

\sectionExercice{Algèbre Linéaire}


\begin{enumerate}
    \item Résolvez le système linéaire suivant :
\[
\begin{cases}
2x + 2y = 4,\\[1mm]
3x - 8y  = -1 
\end{cases}
\]
Réponse :
\[
\begin{cases}
2x + 2y = 4,\\[1mm]
3x - 8y  = -1 
\end{cases}
\Leftrightarrow 
\begin{cases}
2x + 2y = 4,\\[1mm]
3x - 8y  = -1 
\end{cases}
\]
\[
\Leftrightarrow 
\begin{cases}
2x + 2y = 4,\\[1mm]
3x - 8y  = -1 
\end{cases}
\]
\[
\Leftrightarrow 
\begin{cases}
x = 2 - y,\\[1mm]
-11y = -7  
\end{cases}
\]
\[
\Leftrightarrow
\boxed{
\begin{cases}
x = \frac{15}{11}, \\[1mm]
y = \frac{7}{11}
\end{cases}
}
\]\\


    \item Dans \(\mathbb{R}^2\), considérez la base canonique \(\{e_1, e_2\}\) et une nouvelle base \(\{v_1, v_2\}\) définie par
\[
v_1 = \begin{pmatrix} 1 \\ 1 \end{pmatrix}, \quad
v_2 = \begin{pmatrix} 1 \\ -1 \end{pmatrix}.
\]
Exprimez le vecteur 
\[
w = \begin{pmatrix} 2 \\ 3 \end{pmatrix}
\]
dans la nouvelle base.
Réponse :
\[
\lambda_1 v_1 + \lambda_1 v_1 = \begin{pmatrix} 2 \\ 3 \end{pmatrix}
\]
\[
\begin{cases}
\lambda_1 + \lambda_2 = 3,\\[1mm]
\lambda_1 - \lambda_2 = 2 
\end{cases}
\Leftrightarrow 
\begin{cases}
2 \lambda_1 = 5,\\[1mm]
\lambda_1 - \lambda_2 = 2  
\end{cases}
\Leftrightarrow 
\begin{cases}
\lambda_1 = 5/2,\\[1mm]
\lambda_2 = -1/2 
\end{cases}
\]\\
\[
\boxed{
 w = (5/2) v_1 + (-1/2) v_1
}
\]\\


    \item Calculer le produit scalaire des vecteurs $v_1$ et $v_2$ définis ci-dessus.

Réponse :
\[
v_1 . v2 = 1 . 1 + 1 . (-1) \Rightarrow
\boxed{v_1 . v2 = 0}
\]\\


    \item Calculer le déterminant de la matrice:
\[
v_1 = 
    \begin{pmatrix} 
        1 & 2 \\ 
        2 & 3 
    \end{pmatrix}
\]

Réponse :
\[
det(v_1) = 3 - (2 * 2) \Rightarrow
\boxed{det(v_1) = -1}
\]
\end{enumerate}
\end{document}