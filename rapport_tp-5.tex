% !TEX options=-synctex=-1
\documentclass[a4paper,12pt]{article}
\usepackage[utf8]{inputenc}
\usepackage[T1]{fontenc}
\usepackage[french]{babel}
\usepackage{amsmath,amssymb}
\usepackage{xcolor}
\usepackage{geometry}
\usepackage{soul}
\usepackage{graphicx} % pour inclure des images
\usepackage{animate}

\geometry{
  a4paper,
  total={170mm,257mm},
  left=20mm,
  top=20mm,
}

\usepackage{extensionCours}

\DeclareUnicodeCharacter{2212}{-}

\title{TP 5 - Introduction aux quaternions}
\author{Groupe 2 \\* Colin, Henri, Leïla, Loïs}

\begin{document}
\maketitle

\sectionExercice{Exercice 1 - Algèbre des quaternions}
\begin{enumerate}
    \item Soit \( q = 1 + 2i + 3j + 4k \).
          \begin{enumerate}
              \item \emph{Calculez la norme \( \lVert q \rVert \).}
              \begin{equation}
              \lVert q \rVert = \sqrt{1^2 + 2^2 + 3^2 + 4^2}
              \end{equation}
              \begin{equation}
              \boxed{\lVert q \rVert = \sqrt{30} }
              \end{equation}
              \item \emph{Déterminez le conjugué \( \overline{q} \).}
              \begin{equation}
              \boxed{\overline{q}  = 1 - 2i - 3j - 4k }
              \end{equation}
              \item \emph{Vérifiez que \( q \overline{q} = \lVert q \rVert^2 \).} \newline \newline
              Le produit de deux quaternions \( q_1 = (w_1, v_1) \) et \( q_2 = (w_2, v_2) \) est donné par la formule
              \( q_1 q_2 = (w_1 w_2 − v_1 \cdot v_2, w_1 v_2 + w_2 v_1 + v_1 \times v_2) \). \newline
              Ici, on peut écrire \( q = (w, v) \) et \( \overline{q} = (w, -v) \), avec \( w = 1 \) et \( v = (2, 3, 4) \). \newline
              En appliquant la formule de multiplication, on trouve : \newline
              \(q \overline{q} = (w^2 - v \cdot (-v), w (-v) + w v + v \times (-v))\) \newline
              On calcule :
              \begin{equation}
              w^2 = 1
              \end{equation}
              \begin{align}
              v \cdot (-v) &= (2, 3, 4) \cdot (-2, -3, -4) \\
              v \cdot (-v) &= -29
              \end{align}
              \begin{equation}
              w (-v) + w v = 0
              \end{equation}
              \begin{align}
              v \times (-v) &= (2, 3, 4) \times (-2, -3, -4) \\
              v \times (-v) &= 0
              \end{align}
              D'où              
              \begin{align}
               q \overline{q} &= [1 - (-29), 0 ] \\
               q \overline{q} &= 30
              \end{align}
              Donc on a bien :
              \begin{equation}
              \boxed{q \overline{q} = \lVert q \rVert^2}
              \end{equation}
              \item \emph{En déduire l'inverse \( q^{-1} \).} \newline \newline
              L'inverse d'un quaternion \(q\) est donné par la formule \( q^{-1} = \frac{\overline{q}}{\lVert q \rVert^2} \).
              D'où :
              \begin{equation}
              \boxed{q^{-1}  = \frac{1}{30} - \frac{2}{30} i - \frac{3}{30} j - \frac{4}{30} k }
              \end{equation}
          \end{enumerate}
    \item \emph{Soit \( q_1 = 1 + i + j \) et \( q_2 = 1 - j + k \).}
          \begin{enumerate}
              \item \emph{Calculez \( q_1 q_2 \) et \( q_2 q_1 \).} \newline \newline
              On applique la formule :
              \( q_1 q_2 = (w_1 w_2 − v_1 \cdot v_2, w_1 v_2 + w_2 v_1 + v_1 \times v_2) \). \newline
              Ce qui donne :
              \begin{equation}
              q_1 q_2 = [2, (2, -1, 0)]
              \end{equation}
              \begin{equation}
              \boxed{q_1 q_2 = 2 + 2 i - j}
              \end{equation}
              De même on calcule :
              \( q_2 q_1 = (w_1 w_2 − v_2 \cdot v_1, w_1 v_2 + w_2 v_1 + v_2 \times v_1) \). \newline
              Ce qui donne :
              \begin{equation}
              q_2 q_1 = [2, (0, 1, 2)]
              \end{equation}
              \begin{equation}
              \boxed{q_2 q_1 = 2 + j + k}
              \end{equation}
              \item \emph{Vérifiez que la multiplication des quaternions n’est pas commutative.} \newline \newline
              \(q_1 q_2 \neq q_2 q_1\), donc la multiplication des quaternions n'est pas commutative
          \end{enumerate}
\end{enumerate}


\sectionExercice{Modification d'angle}

On souhaite effectuer une rotation de $135^\circ$ autour de l’axe $z$. On considère un vecteur initial $\mathbf{p} = (1, 0, 0)$.

\begin{enumerate}
    \item \textbf{Approche par angles d’Euler}

          On encode cette rotation par les angles d’Euler $(\phi, \theta, \psi) = (0^\circ, 0^\circ, 135^\circ)$ selon l’ordre $ZYX$.
          \begin{enumerate}
              \item \emph{Écrivez les matrices élémentaires $R_x(\phi)$, $R_y(\theta)$ et $R_z(\psi)$.}
              \begin{equation}
              R_x(\phi) = 
              \begin{pmatrix}
              1 & 0 & 0 & 0 \\
              0 & \cos(\phi) & -\sin(\phi) & 0 \\
              0 & \sin(\phi) & \cos(\phi) & 0 \\
              0 & 0 & 0 & 1
              \end{pmatrix}
              \quad \implies \quad
              \boxed{R_x(0) = 
              \begin{pmatrix}
              1 & 0 & 0 & 0 \\
              0 & 1 & 0 & 0 \\
              0 & 0 & 1 & 0 \\
              0 & 0 & 0 & 1
              \end{pmatrix}}
              \end{equation}
              \begin{equation}
              R_y(\theta) = 
              \begin{pmatrix}
              \cos(\theta) & 0 & \sin(\theta) & 0 \\
              0 & 1 & 0 & 0 \\
              -\sin(\theta) & 0 & \cos(\theta) & 0 \\
              0 & 0 & 0 & 1
              \end{pmatrix}
              \quad \implies \quad
              \boxed{R_y(0) = 
              \begin{pmatrix}
              1 & 0 & 0 & 0 \\
              0 & 1 & 0 & 0 \\
              0 & 0 & 1 & 0 \\
              0 & 0 & 0 & 1
              \end{pmatrix}}
              \end{equation}
              Remarquons que \(135^\circ = \frac{3 \pi}{4} \).
              \begin{equation}
              R_z(\psi) = 
              \begin{pmatrix}
              \cos(\psi) & -\sin(\psi) & 0 & 0 \\
              \sin(\psi) & \cos(\psi) & 0 & 0 \\
              0 & 0 & 1 & 0 \\
              0 & 0 & 0 & 1
              \end{pmatrix}
              \quad \implies \quad
              \boxed{R_z(\frac{3 \pi}{4}) = 
              \begin{pmatrix}
              -\frac{\sqrt{2}}{2} & -\frac{\sqrt{2}}{2} & 0 & 0 \\
              \frac{\sqrt{2}}{2} & -\frac{\sqrt{2}}{2} & 0 & 0 \\
              0 & 0 & 1 & 0 \\
              0 & 0 & 0 & 1
              \end{pmatrix}}
              \end{equation}
              \item \emph{Calculez la matrice de rotation totale $R = R_z R_y R_x$.} \newline \newline
              On remarque que \(R_y\) et \(R_x\) correspondent à la matrice identité.
              \begin{align}
              R &= R_z R_y R_x \\
              R &= R_z I_4 I_4 \\ 
              R &= R_z
              \end{align}
              \begin{equation}
              \boxed{R = 
              \begin{pmatrix}
              -\frac{\sqrt{2}}{2} & -\frac{\sqrt{2}}{2} & 0 & 0 \\
              \frac{\sqrt{2}}{2} & -\frac{\sqrt{2}}{2} & 0 & 0 \\
              0 & 0 & 1 & 0 \\
              0 & 0 & 0 & 1
              \end{pmatrix}}
              \end{equation}
              \item \emph{Appliquez cette matrice au vecteur $\mathbf{p}$ pour obtenir $\mathbf{p}_1 = R \mathbf{p}$.}
              \begin{equation}
              p_1 = \begin{pmatrix}
              -\frac{\sqrt{2}}{2} & -\frac{\sqrt{2}}{2} & 0 & 0 \\
              \frac{\sqrt{2}}{2} & -\frac{\sqrt{2}}{2} & 0 & 0 \\
              0 & 0 & 1 & 0 \\
              0 & 0 & 0 & 1
              \end{pmatrix} 
              \begin{pmatrix}
              1 \\ 
              0 \\ 
              0 \\ 
              1
              \end{pmatrix}
              \end{equation}
              \begin{equation}
              \boxed{p_1 = (-\frac{\sqrt{2}}{2}, \frac{\sqrt{2}}{2}, 0)}
              \end{equation}
          \end{enumerate}

    \item \textbf{Approche par quaternion}
          \begin{enumerate}
              \item \emph{Construisez le quaternion unitaire $q = \left[\cos\left(\frac{135^\circ}{2}\right), \sin\left(\frac{135^\circ}{2}\right)\,\mathbf{k}\right]$.}
              \begin{equation}
              \boxed{q = \cos(\frac{3 \pi}{8}) + \cos(\frac{3 \pi}{8}) k}
              \end{equation}
              \item \emph{Écrivez le vecteur initial comme un quaternion pur $p = [0,\,\mathbf{p}]$.}
              \begin{equation}
              \boxed{p = i}
              \end{equation}      
              \item \emph{Calculez $p' = q p q^{-1}$.} \newline \newline
              Le produit de deux quaternions \( q_1 = (w_1, v_1) \) et \( q_2 = (w_2, v_2) \) est donné par la formule
              \( q_1 q_2 = (w_1 w_2 − v_1 \cdot v_2, w_1 v_2 + w_2 v_1 + v_1 \times v_2) \). \newline
              On commence par appliquer cette formule au produit \(pq\). On obtient :
              \begin{equation}
              p q = [0, (\cos(\frac{3 \pi}{8}), \sin(\frac{3 \pi}{8}), 0)]
              \end{equation}
              \begin{equation}
              \boxed{p q = \cos(\frac{3 \pi}{8}) i + \sin(\frac{3 \pi}{8}) j}
              \end{equation}
              L'inverse du quaternion \(q\) est donné par la formule \( q^{-1} = \frac{\overline{q}}{\lVert q \rVert^2} \). Or, \(q\) étant un quaternion unitaire, \(\lVert q \rVert = 1\).
              D'où :
              \begin{align}
              q^{-1}  &= \overline{q} \\
              q^{-1}  &=  \cos(\frac{3 \pi}{8}) - \cos(\frac{3 \pi}{8}) k
              \end{align}
              On applique maintenant la formule de multiplication des quaternions au produit de \(p q\) et \(q^{-1}\). On obtient :
              \begin{equation}
              p q q^{-1} = [0, (\cos(\frac{3 \pi}{8})^2 - \sin(\frac{3 \pi}{8})^2, 2 \cos(\frac{3 \pi}{8}) \sin(\frac{3 \pi}{8}), 0)]
              \end{equation}
              La valeur de \(p' = p q q^{-1} \) est donc :
              \begin{equation}
              \boxed{p' = (\cos(\frac{3 \pi}{8})^2 - \sin(\frac{3 \pi}{8})^2) i + 2 \cos(\frac{3 \pi}{8}) \sin(\frac{3 \pi}{8}) j}
              \end{equation}
              \item \emph{Identifiez la partie vectorielle de $p'$.} \newline \newline 
              $p'$ est un quaternion pur dont la partie vectorielle est : \newline
              \begin{equation}
              \boxed{v = (\cos(\frac{3 \pi}{8})^2 - \sin(\frac{3 \pi}{8})^2) i + 2 \cos(\frac{3 \pi}{8}) \sin(\frac{3 \pi}{8}) j}
              \end{equation}
          \end{enumerate}

    \item \textbf{Comparaison}
          \begin{enumerate}
              \item \emph{Comparez les résultats $\mathbf{p}_1$ et $\mathbf{p}'$. Sont-ils identiques ?} \newline \newline
              On remarque que $p'$ contient des identités trigonométriques remarquables de la forme :
               \begin{equation}
              2 \cos(a) \sin (b) = \sin(a + b) - \sin(a - b)
              \end{equation}
              \begin{equation}
              \cos^2(a) - \sin^2(b) = \cos(a + b) \cos(a - b)
              \end{equation}
              Où \(a = b \) et \(a = \frac{3 \pi}{8} \) \newline 
              On peut donc écrire ces identités sous la forme :
              \begin{equation}
              2 \cos(a) \sin (a) = \sin(2a) - \sin(0)
              \end{equation}
              \begin{equation}
              \cos^2(a) - \sin^2(a) = \cos(2a) \cos(0)
              \end{equation}
              En remplaçant $a$ par sa valeur numérique : 
              \begin{align}
              2 \cos(\frac{3 \pi}{8}) \sin (\frac{3 \pi}{8}) &= \sin(\frac{3 \pi}{4}) - \sin(0) \\
              2 \cos(\frac{3 \pi}{8}) \sin (\frac{3 \pi}{8}) &= \frac{\sqrt{2}}{2}
              \end{align}
              \begin{align}
              \cos^2(\frac{3 \pi}{8}) - \sin^2(\frac{3 \pi}{8}) &= \cos(\frac{3 \pi}{4}) \cos(0) \\
              \cos^2(\frac{3 \pi}{8}) - \sin^2(\frac{3 \pi}{8}) &= -\frac{\sqrt{2}}{2}
              \end{align}
              On en déduit :
              \begin{equation}
                \boxed{p' = -\frac{\sqrt{2}}{2} i + \frac{\sqrt{2}}{2} j}
              \end{equation}
              Ce qui confirme que :
              \begin{equation}
                \boxed{p_1 = p'}
              \end{equation}
              \item \emph{Commentez les avantages de l’approche quaternion par rapport à l’approche Euler dans ce cas particulier.} \newline \newline
              Dans ce cas particulier, l'approche quaternion permet d'éviter le phénomène de blocage de cardan (\emph{gimbal lock}). En effet, lorsque l'on utilise les angles d'Euler pour calculer cette rotation, les matrices de rotation $R_x$ et $R_y$ prennent la valeur d'une matrice identité, ce qui mène à la perte de deux degrés de liberté.
          \end{enumerate}
\end{enumerate}


\sectionExercice{Implémentation Quaternion}
\begin{enumerate}
    \item \emph{Implémenter une classe quaternion.}

    Pour utiliser les quaternions dans notre pipeline de rendu, nous avons implémenté une classe \texttt{Quaternion} représentant un quaternion défini par sa partie scalaire $w$ et les coordonnées de sa partie vectorielle \((x, y, z)\). 
    
    Cette classe possède des méthodes permettant de réaliser les opérations de base sur les quaternions : multiplication (\texttt{prod}), addition (\texttt{add}), calcul de la norme (\texttt{norm}), du conjugué (\texttt{conjugate}) et de l'inverse (\texttt{inverse}).
    Cette nouvelle classe a été ajoutée à notre bibliothèque Mathy dans \texttt{geometry\_engine\_librairie}.

    \item \emph{Implémenter des fonctions pour manipuler vos objects avec des quaternions.}

    Pour calculer des rotations avec des quaternions, nous avons ajouté plusieurs méthodes à la classe \texttt{Quaternion} :
    \begin{itemize}
    \item La méthode \texttt{euler\_to\_quaternion} permet de convertir une rotation exprimée selon les angles d'Euler en quaternion.
    \item La méthode \texttt{to\_euler} permet de convertir une rotation exprimée par un quaternion en angles d'Euler.
    \item La méthode \texttt{to\_rotation\_matrix} permet d'obtenir une matrice de rotation à partir d'un quaternion.
    \end{itemize}

    De plus, pour manipuler nos \texttt{GameObjects} avec des quaternions, nous avons ajouté la méthode \texttt{rotate\_quaternion} à la classe \texttt{transform}. Cette méthode fait appel à \texttt{Quaternion.to\_rotation\_matrix} pour modifier la matrice de transformation associée à un objet.

    \item \emph{Faites tourner un cube avec vos fonctions de quaternions.}
    \item \emph{Afficher un avion dans votre moteur\footnote{Un avion de développeur donc 3 cubes devraient suffire.}.}
\end{enumerate}


\end{document}
