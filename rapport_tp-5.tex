% !TEX options=-synctex=-1
\documentclass[a4paper,12pt]{article}
\usepackage[utf8]{inputenc}
\usepackage[T1]{fontenc}
\usepackage[french]{babel}
\usepackage{amsmath,amssymb}
\usepackage{xcolor}
\usepackage{geometry}
\usepackage{soul}
\usepackage{graphicx} % pour inclure des images
\usepackage{animate}

\geometry{
  a4paper,
  total={170mm,257mm},
  left=20mm,
  top=20mm,
}

\usepackage{extensionCours}

\DeclareUnicodeCharacter{2212}{-}

\title{TP 5 - Introduction aux quaternions}
\author{Groupe 2 \\* Colin, Henri, Leïla, Loïs}

\begin{document}
\maketitle

\sectionExercice{Exercice 1 - Algèbre des quaternions}
\begin{enumerate}
    \item Soit \( q = 1 + 2i + 3j + 4k \).
          \begin{enumerate}
              \item \emph{Calculez la norme \( \lVert q \rVert \).}
              \begin{equation}
              \lVert q \rVert = \sqrt{1^2 + 2^2 + 3^2 + 4^2}
              \end{equation}
              \begin{equation}
              \boxed{\lVert q \rVert = \sqrt{30} }
              \end{equation}
              \item \emph{Déterminez le conjugué \( \overline{q} \).}
              \begin{equation}
              \boxed{\overline{q}  = 1 - 2i - 3j - 4k }
              \end{equation}
              \item \emph{Vérifiez que \( q \overline{q} = \lVert q \rVert^2 \).} \newline \newline
              Le produit de deux quaternions \( q_1 = (w_1, v_1) \) et \( q_2 = (w_2, v_2) \) est donné par la formule
              \( q_1 q_2 = (w_1 w_2 − v_1 \cdot v_2, w_1 v_2 + w_2 v_1 + v_1 \times v_2) \). \newline
              Ici, on peut écrire \( q = (w, v) \) et \( \overline{q} = (w, -v) \), avec \( w = 1 \) et \( v = (2, 3, 4) \). \newline
              En appliquant la formule de multiplication, on trouve : \newline
              \(q \overline{q} = (w^2 - v \cdot (-v), w (-v) + w v + v \times (-v))\) \newline
              On calcule :
              \begin{equation}
              w^2 = 1
              \end{equation}
              \begin{align}
              v \cdot (-v) &= (2, 3, 4) \cdot (-2, -3, -4) \\
              v \cdot (-v) &= -29
              \end{align}
              \begin{equation}
              w (-v) + w v = 0
              \end{equation}
              \begin{align}
              v \times (-v) &= (2, 3, 4) \times (-2, -3, -4) \\
              v \times (-v) &= 0
              \end{align}
              D'où              
              \begin{align}
               q \overline{q} &= (1 - (-29), 0 ) \\
               q \overline{q} &= 30
              \end{align}
              Donc on a bien :
              \begin{equation}
              \boxed{q \overline{q} = \lVert q \rVert^2}
              \end{equation}
              \item \emph{En déduire l'inverse \( q^{-1} \).} \newline \newline
              L'inverse d'un quaternion \(q\) est donné par la formule \( q^{-1} = \frac{\overline{q}}{\lVert q \rVert^2} \).
              D'où :
              \begin{equation}
              \boxed{q^{-1}  = \frac{1}{30} - \frac{2}{30} i - \frac{3}{30} j - \frac{4}{30} k }
              \end{equation}
          \end{enumerate}
    \item \emph{Soit \( q_1 = 1 + i + j \) et \( q_2 = 1 - j + k \).}
          \begin{enumerate}
              \item \emph{Calculez \( q_1 q_2 \) et \( q_2 q_1 \).} \newline \newline
              On applique la formule :
              \( q_1 q_2 = (w_1 w_2 − v_1 \cdot v_2, w_1 v_2 + w_2 v_1 + v_1 \times v_2) \). \newline
              Ce qui donne :
              \begin{equation}
              q_1 q_2 = (2, (2, -1, 0))
              \end{equation}
              \begin{equation}
              \boxed{q_1 q_2 = 2 + 2 i - j}
              \end{equation}
              De même on calcule :
              \( q_2 q_1 = (w_1 w_2 − v_2 \cdot v_1, w_1 v_2 + w_2 v_1 + v_2 \times v_1) \). \newline
              Ce qui donne :
              \begin{equation}
              q_2 q_1 = (2, (0, 1, 2))
              \end{equation}
              \begin{equation}
              \boxed{q_2 q_1 = 2 + j + k}
              \end{equation}
              \item \emph{Vérifiez que la multiplication des quaternions n’est pas commutative.} \newline \newline
              \(q_1 q_2 \neq q_2 q_1\), donc la multiplication des quaternions n'est pas commutative
          \end{enumerate}
\end{enumerate}


\sectionExercice{Modification d'angle}

On souhaite effectuer une rotation de $135^\circ$ autour de l’axe $z$. On considère un vecteur initial $\mathbf{p} = (1, 0, 0)$.

\begin{enumerate}
    \item \textbf{Approche par angles d’Euler}

          On encode cette rotation par les angles d’Euler $(\phi, \theta, \psi) = (0^\circ, 0^\circ, 135^\circ)$ selon l’ordre $ZYX$.
          \begin{enumerate}
              \item \emph{Écrivez les matrices élémentaires $R_x(\phi)$, $R_y(\theta)$ et $R_z(\psi)$.}
              \item Calculez la matrice de rotation totale $R = R_z R_y R_x$.
              \item Appliquez cette matrice au vecteur $\mathbf{p}$ pour obtenir $\mathbf{p}_1 = R \mathbf{p}$.
          \end{enumerate}

    \item \textbf{Approche par quaternion}
          \begin{enumerate}
              \item Construisez le quaternion unitaire $q = \left[\cos\left(\frac{135^\circ}{2}\right), \sin\left(\frac{135^\circ}{2}\right)\,\mathbf{k}\right]$.
              \item Écrivez le vecteur initial comme un quaternion pur $p = [0,\,\mathbf{p}]$.
              \item Calculez $p' = q p q^{-1}$.
              \item Identifiez la partie vectorielle de $p'$.
          \end{enumerate}

    \item \textbf{Comparaison}
          \begin{enumerate}
              \item Comparez les résultats $\mathbf{p}_1$ et $\mathbf{p}'$. Sont-ils identiques ?
              \item Commentez les avantages de l’approche quaternion par rapport à l’approche Euler dans ce cas particulier.
          \end{enumerate}
\end{enumerate}


\sectionExercice{Implémentation Quaternion}
\begin{enumerate}
    \item Implémenter une classe quaternion.
    \item Implémenter des fonctions pour manipuler vos objects avec des quaternions.
    \item Faites tourner un cube avec vos fonctions de quaternions.
    \item Afficher un avion dans votre moteur\footnote{Un avion de développeur donc 3 cubes devraient suffire.}.
\end{enumerate}


\end{document}
