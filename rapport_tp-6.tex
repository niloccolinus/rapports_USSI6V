% !TEX options=-synctex=-1
\documentclass[a4paper,12pt]{article}
\usepackage[utf8]{inputenc}
\usepackage[T1]{fontenc}
\usepackage[french]{babel}
\usepackage{amsmath,amssymb}
\usepackage{xcolor}
\usepackage{geometry}
\usepackage{soul}
\usepackage{graphicx} % pour inclure des images
\usepackage{animate}

\geometry{
  a4paper,
  total={170mm,257mm},
  left=20mm,
  top=20mm,
}

\usepackage{extensionCours}

\DeclareUnicodeCharacter{2212}{-}

\title{TP 6 - Blocage de Cardan et Interpolation}
\author{Groupe 2 \\* Colin, Henri, Leïla, Loïs}

\begin{document}
\maketitle

\sectionExercice{Exercice 1 - Référentiel local vs global}

Soit un vecteur \( \mathbf{p} = (1, 0, 0) \). On considère deux rotations successives :
\begin{itemize}
  \item une rotation \( \theta = 90^\circ \) autour de \( \mathbf{z} \) dans le repère \textbf{global},
  \item suivie d’une rotation \( \alpha = 90^\circ \) autour de \( \mathbf{y} \) dans le repère \textbf{local} (lié au vecteur déjà tourné).
\end{itemize}

\begin{enumerate}
          \item \emph{Construisez les quaternions \( q_1 \) (rotation autour de \( \mathbf{z} \)) et \( q_2 \) (rotation locale autour de \( \mathbf{y} \)).}

            Le quaternion \( q_1 \) se construit par la combinaison des trois quaternions élémentaires :

            \[
              q_{x1}(0) = (1, 0i) \newline
            \]
            \[
              q_{y1}(0) = (1, 0j) \newline  
            \]
            \[ 
              q_{z1}(\theta) 
              = \left(cos \left( \frac{\theta}{2} \right), sin \left(\frac{\theta}{2} \right)k\right) 
              = \left(cos \left( \frac{\pi}{4} \right), sin \left(\frac{\pi}{4} \right)k\right) \newline 
              = \left( \frac{\sqrt{2}}{2}, \frac{\sqrt{2}}{2}k \right) 
            \]

            On trouve :

            \[
              q_1 = q_{z1}(\theta) q_{y1}(0) q_{x1}(0) = \boxed{\frac{\sqrt{2}}{2} + \frac{\sqrt{2}}{2}k}
            \]

            Le quaternion \( q_2 \) se construit par la combinaison des trois quaternions élémentaires :

            \[
              q_{x2}(0) = (1, 0i) \newline
            \]
            \[ 
              q_{y2}(\alpha) 
              = \left(cos \left( \frac{\alpha}{2} \right), sin \left(\frac{\alpha}{2} \right)j\right) 
              = \left(cos \left( \frac{\pi}{4} \right), sin \left(\frac{\pi}{4} \right)j\right) \newline 
              = \left( \frac{\sqrt{2}}{2}, \frac{\sqrt{2}}{2}j \right) 
            \]
            \[
              q_{z2}(0) = (1, 0k) \newline  
            \]

            On trouve :

            \[
              q_1 = q_{z2}(0) q_{y2}(\alpha) q_{x2}(0) = \boxed{\frac{\sqrt{2}}{2} + \frac{\sqrt{2}}{2}j}
            \]

          \newpage \item \emph{Donnez le quaternion global \( q_{\text{final}} \) à appliquer sur \( \mathbf{p} \), en respectant l’ordre des référentiels.} 

            La rotation globale \( q_1 \) s'applique en premier, donc on applique la multiplication dans l'ordre :

            \[
              q_{final} = q_2 q_1
            \]

            Le produit de deux quaternions \( q_2 = (w_2, v_2) \) et \( q_1 = (w_1, v_1) \) est donné par la formule
            \( q_2 q_1 = (w_2 w_1 − v_2 \cdot v_1, w_2 v_1 + w_1 v_2 + v_2 \wedge v_1) \). \newline

            On calcule d'abord la partie scalaire (réelle) :

            \[ w_2 w_1 = \frac{\sqrt{2}}{2} \times \frac{\sqrt{2}}{2} = \frac{1}{2}\]
            \[ v_2 \cdot v_1 = 
            \begin{pmatrix} 0 \\ \frac{\sqrt{2}}{2} \\ 0 \end{pmatrix}
            \cdot
            \begin{pmatrix} 0 \\ 0 \\ \frac{\sqrt{2}}{2} \end{pmatrix}
            = \begin{pmatrix} 0 \\ 0 \\ 0 \end{pmatrix}
            \]

            La partie scalaire vaut alors :

            \[
               w_2 w_1 - v_2 \cdot v_1 = \frac{1}{2}
            \]

            On calcule maintenant la partie vectorielle (imaginaire) :

            \[ w_2 v_1 = \frac{\sqrt{2}}{2} \times \begin{pmatrix} 0 \\ 0 \\ \frac{\sqrt{2}}{2} \end{pmatrix}
            = \begin{pmatrix} 0 \\ 0 \\ \frac{1}{2} \end{pmatrix} \]
            \[ w_1 v_2 = \frac{\sqrt{2}}{2} \times \begin{pmatrix} 0 \\ \frac{\sqrt{2}}{2} \\ 0 \end{pmatrix}
            = \begin{pmatrix} 0 \\ \frac{1}{2} \\ 0 \end{pmatrix} \]
            \[ v_2 \wedge v_1 
            = \begin{pmatrix} 0 \\ 0 \\ \frac{\sqrt{2}}{2} \end{pmatrix}
            \wedge \begin{pmatrix} 0 \\ \frac{\sqrt{2}}{2} \\ 0 \end{pmatrix} 
            = \begin{pmatrix} \frac{1}{2} \\ 0 \\ 0 \end{pmatrix}
            \]

            La partie vectorielle vaut alors :

            \[ w_2 v_1 + w_1 v_2 + v_2 \wedge v_1 = 
              \begin{pmatrix} 0 \\ 0 \\ \frac{1}{2} \end{pmatrix} + 
              \begin{pmatrix} 0 \\ \frac{1}{2} \\ 0 \end{pmatrix} + 
              \begin{pmatrix} \frac{1}{2} \\ 0 \\ 0 \end{pmatrix} = 
              \begin{pmatrix} \frac{1}{2} \\ \frac{1}{2} \\ \frac{1}{2} \end{pmatrix}
            \]

            Le quaternion \( q_{final} \) vaut donc :

            \[ \boxed{q_{final} = \frac{1}{2} + \frac{1}{2}i + \frac{1}{2}j + \frac{1}{2}k} \]
              
          \item \emph{Calculez \( p' = q_{\text{final}}\,p\,q_{\text{final}}^{-1} \).}

            On calcule d'abord l'inverse :

            \[ q_{\text{final}}^{-1} = \frac{\overline{q_{final}}}{{\lVert q \rVert}^2} \]

            \newpage où

            \[ \overline{q_{final}} = \frac{1}{2} - \frac{1}{2}i - \frac{1}{2}j - \frac{1}{2}k \]

            et 

            \[ {\lVert q \rVert}^2 = \frac{1}{2}^2 + \frac{1}{2}^2 + \frac{1}{2}^2 + \frac{1}{2}^2 = 1 \]

            Donc :

            \[ q_{\text{final}}^{-1} = \frac{1}{2} - \frac{1}{2}i - \frac{1}{2}j - \frac{1}{2}k \]

            On calcule ensuite \(q_{final} \times p\).

            Le vecteur \( p \) peut s'écrire sous la forme d'un quaternion :

            \[ p_q = 0 + 1i + 0j + 0k = i \]

            \[ 
              q_{\text{final}} \times p_q = 
              \left(  \frac{1}{2} + \frac{1}{2}i - \frac{1}{2}j - \frac{1}{2}k \right) \times i
              = -\frac{1}{2} + \frac{1}{2}i + \frac{1}{2}j - \frac{1}{2}k
            \]

            On peut maintenant calculer \( p' \). On note :

            \[ p' = q_a q_b \]

            où :

            \[ q_a = q_{\text{final}} \times p_q = (w_a, v_a) \] 


            et 

            \[ q_b = q_{\text{final}}^{-1} = (w_b, v_b) \]

            La multiplication s'écrit :

            \[ q_a q_b = (w_a w_b - v_a \cdot v_b, w_b v_a + w_a v_b + v_a \wedge v_b) \]

            On calcule d'abord la partie scalaire :

            \[ w_a w_b = -\frac{1}{2} \times \frac{1}{2} = -\frac{1}{4} \]

            \[ 
              v_a \cdot v_b =
              \begin{pmatrix} \frac{1}{2} \\ \frac{1}{2} \\ -\frac{1}{2} \end{pmatrix} \cdot
              \begin{pmatrix} -\frac{1}{2} \\ -\frac{1}{2} \\ -\frac{1}{2} \end{pmatrix} =
              \begin{pmatrix} -\frac{1}{4} \\ -\frac{1}{4} \\ \frac{1}{4} \end{pmatrix}
            \]

            La partie scalaire vaut :

            \[ 
              w_a w_b - v_a \cdot v_b =
              -\frac{1}{4} - \left( -\frac{1}{4} -\frac{1}{4} + \frac{1}{4} \right) 
              = 0
            \]

            \newpage On calcule maintenant la partie vectorielle :

            \[ 
              w_a v_b = -\frac{1}{2} \times \left( -\frac{1}{2}, -\frac{1}{2}, -\frac{1}{2} \right) 
              = \left( \frac{1}{4}, \frac{1}{4}, \frac{1}{4} \right) 
            \]

            \[ 
              w_b v_a = \frac{1}{2} \times \left( \frac{1}{2}, \frac{1}{2}, -\frac{1}{2} \right) 
              = \left( \frac{1}{4}, \frac{1}{4}, -\frac{1}{4} \right) 
            \]

            \[
              v_a \wedge v_b = 
              \begin{pmatrix} \frac{1}{2} \\ \frac{1}{2} \\ -\frac{1}{2} \end{pmatrix} 
              \wedge 
              \begin{pmatrix} -\frac{1}{2} \\ -\frac{1}{2} \\ -\frac{1}{2} \end{pmatrix} 
              = 
              \begin{pmatrix} -\frac{1}{2} \\ \frac{1}{2} \\ 0 \end{pmatrix}
            \]

            La partie vectorielle vaut :

            \[
               w_a v_b + w_b v_a + v_a \wedge v_b = 
              \begin{pmatrix} \frac{1}{4} \\ \frac{1}{4} \\ \frac{1}{4} \end{pmatrix} +
              \begin{pmatrix} \frac{1}{4} \\ \frac{1}{4} \\ -\frac{1}{4} \end{pmatrix} +
              \begin{pmatrix} -\frac{1}{2} \\ \frac{1}{2} \\ 0 \end{pmatrix} =
              \begin{pmatrix} 0 \\ 1 \\ 0 \end{pmatrix}
            \]

            Le quaternion \( p' \) vaut donc :

            \[ p' = 0 + 0i + 1j + 0k = \boxed{j} \] 


          \item \emph{Donnez les coordonnées finales du vecteur tourné.}

            Les coordonnées finales sont : \[ (0, 1, 0) \]

          \item \emph{Que se passe-t-il si les deux rotations sont appliquées dans le même repère (global ou local uniquement) ?}

          Au lieu de construire deux quaternions \( q_1 \) et \( q_2 \), on construit un seul quaternion global avec les trois quaternions élémentaires :

            \[
              q_{x}(0) = (1, 0i) \newline
            \]
            \[ 
              q_{y}(\alpha) 
              = \left(cos \left( \frac{\alpha}{2} \right), sin \left(\frac{\alpha}{2} \right)j\right) 
              = \left( \frac{\sqrt{2}}{2}, \frac{\sqrt{2}}{2}j \right) 
            \]
            \[ 
              q_{z}(\theta) 
              = \left(cos \left( \frac{\theta}{2} \right), sin \left(\frac{\theta}{2} \right)k\right) 
              = \left( \frac{\sqrt{2}}{2}, \frac{\sqrt{2}}{2}k \right) 
            \]

            \[ q_{global} = q_{z}(\theta) q_{y}(\alpha) q_{x}(0) \]

            Avec la même méthode de multiplication utilisée précédemment on trouve :

            \[ q_{global} = \boxed{\frac{1}{2} + \frac{1}{2}i + \frac{1}{2}j + \frac{1}{2}k} \]

            On constate que \(q_{global} = q_{final}\). Cela montre que la composition d'une rotation globale suivie d'une rotation locale calculée avec \(q_{final}\) peut aussi s'exprimer comme une unique rotation dans le repère global (ici \(q_{global}\)).

          \newpage \item \emph{Que remarque-t-on si l’on inverse l’ordre de multiplication \( q_1 \cdot q_2 \) au lieu de \( q_2 \cdot q_1 \) ?}

            En calculant \(q_1 q_2\) avec la méthode précédente on trouve :

            \[
              q_1 q_2 = \boxed{\frac{1}{2} -\frac{1}{2}i + \frac{1}{2}j + \frac{1}{2}k}
            \]

            On remarque que \(q_1 q_2 \neq q_2 q_1\). Cela montre que l'ordre de multiplication de deux quaternions n'est pas commutatif.
            Ainsi, si l’on décide d’appliquer une rotation locale avant la rotation globale, alors on ne peut plus regrouper ces deux rotations en un unique quaternion exprimé dans le repère global. 


\end{enumerate}

\end{document}
