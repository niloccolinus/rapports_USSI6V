\documentclass[a4 paper, 12pt]{article}
\usepackage[utf8]{inputenc}
\usepackage[T1]{fontenc}
\usepackage[french]{babel}
\title{Rapport TP 0}
\author{Colin \and Henri \and Leïla \and Loïs}
\date{}

\begin{document}
\maketitle

Ce premier TP consiste à installer l'environnement de développement nécessaire au projet de moteur de géométrie : Git, Sublime Text, Python, structure de projet sur Github.

Nous n'avons pas encore choisi de nom pour notre moteur de géométrie, nos repos qui y font référence commencent donc juste par "nom moteur" pour le moment.

Pour prendre en main l'IDE Sublime Text, nous avons créé un fichier python intitulé tp\_0\_ins.py qui affiche "Hello World" quand on l'exécute. Grâce au fichier geometry\_engine.sublime-project qui l'accompagne, il suffit d'appuyer sur Ctrl + Shift + B afin de le lancer automatiquement dans Sublime Text.

Les fichiers tp\_0\_ins.py et geometry\_engine.sublime-project sont trouvables dans le sous-module nom\_moteur\_librairie.

\end{document}
