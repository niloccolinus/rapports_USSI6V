% !TEX options=-synctex=-1
\documentclass[a4paper,12pt]{article}
\usepackage[utf8]{inputenc}
\usepackage[T1]{fontenc}
\usepackage[french]{babel}
\usepackage{amsmath,amssymb}
\usepackage{xcolor}
\usepackage{geometry}
\usepackage{tikz}
\usepackage{soul}

\geometry{
  a4paper,
  total={170mm,257mm},
  left=20mm,
  top=20mm,
}


\usepackage{extensionCours}

\title{TP 2 - Géométrie Classique et Triangulation de Delaunay}
\author{Groupe 2 \\* Colin, Henri, Leïla, Loïs}

\begin{document}
\maketitle

\sectionExercice{Algèbre Linéaire}


\begin{enumerate}
   \item On définit le cercle $\mathcal{C}_1$ de diamètre le segment $[AB]$, montrer que le point $C$ appartient à $\mathcal{C}_1$, 

Réponse :

  Le cercle $\mathcal{C}_1$ a pour diamètre le segment $[AB]$. Soit $D$ le centre de ce cercle est le milieu de $[AB]$, donc $D = \begin{pmatrix}2 & 0\end{pmatrix}$.

  Donc le rayon du cercle $\mathcal{C}_1$ est $2$.

  Calculons la distance entre $C$ et $D$ :

  \[
  \text{Distance}(C, D) = \sqrt{(2-2)^2 + (2-0)^2} = \sqrt{0 + 4} = \sqrt{4} = 2
  \]
  \[
\boxed{
  \text{Distance}(C, D) = 2}
  \]

  Cette distance étant égale au rayon, alors $C$ appartient à $\mathcal{C}_1$.
\newline
   \item En utilisant le théorème de Pythagore montrer que le triangle $ABC$ est rectangle. 

  Réponse :

  Afin d'utilser le théorème de Pythagore, calculons les longeurs des côtés du triangle $ABC$:

  \begin{itemize}
    	\item $AB = \sqrt{(4-0)^2 + (0-0)^2} = 4$
    	\item $AC = \sqrt{(2-0)^2 + (2-0)^2} = \sqrt{4 + 4} = \sqrt{8} = 2\sqrt{2}$
    	\item $BC = \sqrt{(2-4)^2 + (2-0)^2} = \sqrt{4 + 4} = \sqrt{8} = 2\sqrt{2}$
  \end{itemize}

  Vérifions ensuite l'égalité suivante :
  \[
  AB^2 = AC^2 + BC^2 \implies 4^2 = (2\sqrt{2})^2 + (2\sqrt{2})^2 \implies 16 = 8 + 8 \implies 16 = 16
  \]
  \[
  \boxed{
  AB^2 = AC^2 + BC^2
  }
  \]

  Puisque l'égalité est vérifiée, alors d'après le théorème de Pythagore le triangle $ABC$ est rectangle en $C$.
\newpage
   \item Soit $D$ le centre de $\mathcal{C}_1$, soit $\Delta$ la bissectrice de l'angle $\widehat{CDB}$, on note $E$ l'intersection entre $\mathcal{C}_1$ et $\Delta$. En déduire la longueur du segment $[DE]$. 

Réponse :

  Pour commencer, déterminons les coordonnées du point $E$.

  La bissectrice $\Delta$ de l'angle $\widehat{CDB}$ est la droite qui passe par $D$ et qui fait un angle de $45^\circ$ avec l'axe des abscisses car CD est la bissectrice de $\widehat{ACB}$ donc $\widehat{CDB}$ est un angle rectangle.

  L'équation de la droite $\Delta$ est donc $y = x - 2$.

  Le point $E$ est l'intersection entre $\mathcal{C}_1$ et $\Delta$. L'équation du cercle $\mathcal{C}_1$ est :

  \[
  (x-D_x)^2 + (y-D_y)^2 = 4
  \]
  \[
  (x-2)^2 + y^2 = 4
  \]

  En injectant l'équation de $\Delta$ dans l'équation du cercle $\mathcal{C}_1$, nous obtenons :

  \[
  (x-2)^2 + (x-2)^2 = 4 \implies 2(x-2)^2 = 4 \implies (x-2)^2 = 2 \implies x-2 = \pm \sqrt{2} \implies x = 2 \pm \sqrt{2}
  \]
  \[
  \boxed{
   x = 2 \pm \sqrt{2}
  }
  \]

  Les coordonnées de $E$ sont donc $(2 + \sqrt{2}, \sqrt{2})$ ou $(2 - \sqrt{2}, -\sqrt{2})$. Nous choisissons $E = (2 + \sqrt{2}, \sqrt{2})$.

  Puis calculons la distance $DE$ :

  \[
  \text{Distance}(D, E) = \sqrt{(2 + \sqrt{2} - 2)^2 + (\sqrt{2} - 0)^2} = \sqrt{(\sqrt{2})^2 + (\sqrt{2})^2} = \sqrt{2 + 2} = \sqrt{4} = 2
  \]
  \[
  \boxed{
  \text{Distance}(D, E) = 2
  }
  \]
\newline
   \item Soit $\mathcal{C}_2$ le cercle de centre $E$ et de rayon $[DE]$. Par définition $D$ est la première intersection entre $\mathcal{C}_2$ et $\Delta$, on note $F$ la seconde intersection entre $\mathcal{C}_2$ et $\Delta$. Donnez la longueur du segment $[DF]$. 

Réponse :

  Le cercle $\mathcal{C}_2$ a pour centre $E = \left(2 + \sqrt{2}, \sqrt{2} \right)$ et pour rayon $[DE] = 2$. Le point $F$ est la seconde intersection entre le cercle $\mathcal{C}_2$ et la bissectrice $\Delta$ d'équation $y = x - 2$.

  \medskip

  En injectant cette équation dans celle du cercle :
  \[
  (x - (2 + \sqrt{2}))^2 + (y - \sqrt{2})^2 = 4,
  \]
  on obtient :
  \[
  (x - 2 - \sqrt{2})^2 + (x - 2 - \sqrt{2})^2 = 4.
  \]
  \[
  2(x - 2 - \sqrt{2})^2 = 4 \Rightarrow (x - 2 - \sqrt{2})^2 = 2 \Rightarrow x - 2 - \sqrt{2} = \pm \sqrt{2}.
  \]

  Les solutions sont :
  \[
  \boxed{x = 2,}\quad \text{et} \quad \boxed{x = 2 + 2\sqrt{2}}.
  \]

  Le cas $x = 2$ correspond au point $D$, donc :
  \[
  \boxed{
  F = \left(2 + 2\sqrt{2},\ 2\sqrt{2} \right).
  }
  \]

  Calculons la distance $DF$ :
  \[
  \text{Distance}(D, F) = \sqrt{(2 + 2\sqrt{2} - 2)^2 + (2\sqrt{2} - 0)^2} = \sqrt{(2\sqrt{2})^2 + (2\sqrt{2})^2} = \sqrt{8 + 8} = \sqrt{16} = 4.
  \]
  \[
  \boxed{
  \text{Distance}(D, F) = 4
  }
  \]
\newline
   \item On note $d$ la droite formée par les points $[BE]$, soit $d'$ la droite définie comme l'unique droite parallèle à $d$ et passant par le point $F$. On note $G$ l'intersection entre la droite $d'$ et l'axe engendré par $\vec{e_1}$. En utilisant le théorème de Thalès, déterminez les coordonnées du point $G$. 

  Réponse :

La droite \( d \) passe par \( B \) et \( E \). Sa pente est :
\[
m = \frac{y_E - y_B}{x_E - x_B} = \frac{\sqrt{2}}{(2 + \sqrt{2}) - 4} = \frac{\sqrt{2}}{-2 + \sqrt{2}} = -\sqrt{2} - 1
\]

La droite \( d' \) est parallèle à \( d \) et passe par \( F \). Elle a donc la même pente, soit :
\[
y - y_F = m(x - x_F) \quad \text{avec } F = (2 + 2\sqrt{2}, 2\sqrt{2})
\]

On cherche les coordonnées de \( G \), point d'intersection de \( d' \) avec l'axe des abscisses \( y = 0 \).

On remplace \( y = 0 \) dans l'équation de la droite \( d' \) :
\begin{align*}
0 - 2\sqrt{2} &= (-\sqrt{2} - 1)(x - (2 + 2\sqrt{2})) \\
\frac{-2\sqrt{2}}{-\sqrt{2} - 1} &= x - (2 + 2\sqrt{2})
\end{align*}

On rationalise :
\[
\frac{-2\sqrt{2}}{-\sqrt{2} - 1} = \frac{2\sqrt{2}}{\sqrt{2} + 1} = \frac{2\sqrt{2}(\sqrt{2} - 1)}{(\sqrt{2} + 1)(\sqrt{2} - 1)} = \frac{2\sqrt{2}(\sqrt{2} - 1)}{1} = 2(2 - \sqrt{2}) = 4 - 2\sqrt{2}
\]

Puis :
\[
x = (4 - 2\sqrt{2}) + (2 + 2\sqrt{2}) = 6
\]

\[
\boxed{G = (6,\ 0)}
\]
\newpage
   \item Représenter la figure obtenue à partir des différentes constructions ci-dessus. Vous pouvez utiliser le package \href{https://www.overleaf.com/learn/latex/TikZ_package}{TikZ} qui permet de générer des figures vectorisées dans le style du LaTex

Réponse :
\newline
\begin{tikzpicture}[scale=1.5]
  % Coordonnées intermédiaires
  \def\sqrttwo{1.4142}
  \coordinate (A) at (0,0);
  \coordinate (B) at (4,0);
  \coordinate (C) at (2,2);
  \coordinate (D) at (2,0);
  \coordinate (E) at ({2 + \sqrttwo}, {\sqrttwo});
  \coordinate (F) at ({2 + 2*\sqrttwo}, {2*\sqrttwo});

  % Cercle C1 de diamètre AB
  \draw[dashed, gray] (D) circle (2);

  % Cercle C2 de centre E, rayon DE = 2
  \draw[dashed, gray] (E) circle (2);

  % Triangle ABC
  \draw[thick] (A) -- (B) -- (C) -- cycle;

  % Segments clés
  \draw[blue] (D) -- (C);
  \draw[blue] (D) -- (B);
  \draw[blue] (D) -- (E); % bissectrice
  \draw[blue] (D) -- (F);

  % Droite d = (BE)
  \draw[red] (B) -- (E);

  % Calcul de la pente de la droite BE
  \pgfmathsetmacro{\slopeBE}{(\sqrttwo - 0)/(2 + \sqrttwo - 4)}

  % Intersection avec l'axe x => point G
  \pgfmathsetmacro{\xG}{2 + 2*\sqrttwo - (2*\sqrttwo)/\slopeBE}
  \coordinate (G) at (\xG, 0);

% Droite d' = (FG)
  \draw[red] (F) -- (G);

  % Axes
  \draw[->, gray] (-1,0) -- (7,0) node[right] {$x$};
  \draw[->, gray] (0,-1) -- (0,5) node[above] {$y$};

  % Points
  \foreach \point/\name in {A/A, B/B, C/C, D/D, E/E, F/F, G/G}
    \filldraw (\point) circle (2pt) node[above right] {\name};

\end{tikzpicture}
\newline
\end{enumerate}
\end{document}