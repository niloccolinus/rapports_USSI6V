% !TEX options=-synctex=-1
\documentclass[a4paper,12pt]{article}
\usepackage[utf8]{inputenc}
\usepackage[T1]{fontenc}
\usepackage[french]{babel}
\usepackage{amsmath,amssymb}
\usepackage{xcolor}
\usepackage{geometry}
\usepackage{soul}

\geometry{
  a4paper,
  total={170mm,257mm},
  left=20mm,
  top=20mm,
}


\usepackage{extensionCours}

\title{TP 2 - Géométrie Classique et Triangulation de Delaunay}
\author{Groupe 2 \\* Colin, Henri, Leïla, Loïs}

\begin{document}
\maketitle

\sectionExercice{Algèbre Linéaire}


\begin{enumerate}
   \item On définit le cercle $\mathcal{C}_1$ de diamètre le segment $[AB]$, montrer que le point $C$ appartient à $\mathcal{C}_1$, 

Réponse :
\newline
   \item En utilisant le théorème de Pythagore montrer que le triangle $ABC$ est rectangle. 

  Réponse :
\newline
   \item Soit $D$ le centre de $\mathcal{C}_1$, soit $\Delta$ la bissectrice de l'angle $\widehat{CDB}$, on note $E$ l'intersection entre $\mathcal{C}_1$ et $\Delta$. En déduire la longueur du segment $[DE]$. 

Réponse :
\newline
   \item Soit $\mathcal{C}_2$ le cercle de centre $E$ et de rayon $[DE]$. Par définition $D$ est la première intersection entre $\mathcal{C}_2$ et $\Delta$, on note $F$ la seconde intersection entre $\mathcal{C}_2$ et $\Delta$. Donnez la longueur du segment $[DF]$. 

Réponse :
\newline
   \item On note $d$ la droite formée par les points $[BE]$, soit $d'$ la droite définie comme l'unique droite parallèle à $d$ et passant par le point $F$. On note $G$ l'intersection entre la droite $d'$ et l'axe engendré par $\vec{e_1}$. En utilisant le théorème de Thalès, déterminez les coordonnées du point $G$. 

  Réponse :
\newline
   \item Représenter la figure obtenue à partir des différentes constructions ci-dessus. Vous pouvez utiliser le package \href{https://www.overleaf.com/learn/latex/TikZ_package}{TikZ} qui permet de générer des figures vectorisées dans le style du LaTex
\footnote{Vous pouvez vous aider de l'éditeur en ligne \href{https://www.mathcha.io/}{Mathcha} est un éditeur WYSIWYG qui permet de directement visualiser et de générer du code TikZ.}.  

Réponse :
\newline
\end{enumerate}
\end{document}