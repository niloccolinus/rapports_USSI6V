% !TEX options=-synctex=-1
\documentclass[a4paper,12pt]{article}
\usepackage[utf8]{inputenc}
\usepackage[T1]{fontenc}
\usepackage[french]{babel}
\usepackage{amsmath,amssymb}
\usepackage{xcolor}
\usepackage{geometry}
\usepackage{soul}
\usepackage{graphicx} % pour inclure des images

\geometry{
  a4paper,
  total={170mm,257mm},
  left=20mm,
  top=20mm,
}

\usepackage{extensionCours}

\title{TP 3 - Similitudes}
\author{Groupe 2 \\* Colin, Henri, Leïla, Loïs}

\begin{document}
\maketitle

\sectionExercice{Théorie- Un bug dans Unity}
Vous êtes développeur d'un jeu sous Unity. Malheureusement, comme à son habitude le logiciel est défectueux et l'éditeur ne fonctionne plus du tout, de même que les fonctions qui permettent de définir la position et l'orientation manuellement. 
\begin{enumerate}
    \item Soit un personnage modélisé par un sprite instancié à l'origine du moteur, l'utilisateur souhaite le déplacer à la position $\begin{pmatrix} 1 & 2\end{pmatrix}$, donner la matrice de changement de coordonnée à appliquer au personnage pour lui donner sa nouvelle position.\label{exo:1.1}\newline \newline
Réponse :

On cherche la matrice de transpositions correspondant aux coordonnées homogènes mondes du personnage et nommée $T_p$ : 

\[
T_p = 
\begin{pmatrix}
1 & 0 & a \\
0 & 1 & b \\
0 & 0 & 1
\end{pmatrix}\]

Tel que : 
\[\begin{pmatrix}
1 & 0 & a \\
0 & 1 & b \\
0 & 0 & 1
\end{pmatrix}
\begin{pmatrix}
0 \\
0 \\
1 
\end{pmatrix}
=
\begin{pmatrix}
1 \\
2 \\
1
\end{pmatrix}\]

Donc la matrice de transposition $T_p$ du personnage est égale à :
\[\boxed{
T_p = 
\begin{pmatrix}
1 & 0 & 1 \\
0 & 1 & 2 \\
0 & 0 & 1
\end{pmatrix}}\] \newline

    \item L'utilisateur souhaite également faire tourner le personnage de \(\frac{\pi}{4}\) dans le sens trigonométrique autour de l'origine. Donnez la matrice de rotation correspondante en coordonnées homogènes. \label{exo:1.2}\newline \newline
Réponse :

On cherche la matrice de rotations correspondant aux coordonnées homogènes mondes du personnage et nommée $R_p$ : 
\[
R_p = 
\begin{pmatrix}
\cos(\theta) & -\sin(\theta) & 0 \\
\sin(\theta) & \cos(\theta) & 0 \\
0 & 0 & 1
\end{pmatrix}\]

Ici \(\theta\) = \(\frac{\pi}{4}\) donc : \newline
\[\boxed{
R_p = 
\begin{pmatrix}
\frac{\sqrt{2}}{2} & \frac{-\sqrt{2}}{2} & 0 \\
\frac{\sqrt{2}}{2} & \frac{\sqrt{2}}{2} & 0 \\
0 & 0 & 1
\end{pmatrix}}\] \newline
    \item On souhaite doubler la taille du personnage. Donnez la matrice d'homothétie correspondante.\newline \newline
Réponse :

On cherche la matrice d'homotétie correspondant aux coordonnées homogènes mondes du personnage et nommée $H_p$ : 
\[
H_p = 
\begin{pmatrix}
k & 0 & 0 \\
0 & k & 0 \\
0 & 0 & 1
\end{pmatrix}\]

Ici $k$ = 2 donc : 
\[\boxed{
H_p = 
\begin{pmatrix}
2 & 0 & 0 \\
0 & 2 & 0 \\
0 & 0 & 1
\end{pmatrix}}\] \newline

    \item Considérez l'épée, qui doit être placée dans le repère local du personnage à la position $\begin{pmatrix} 0.5 & -0.5 \end{pmatrix}$. De plus l'épée est tournée de $\pi$ par rapport au personnage. Définissez la matrice de transformation locale à appliquer à l'épée par rapport au repère du personnage.\newline \newline
Réponse :

On cherche les matrices de translation $T_e$ et de rotation $R_e$ correspondant aux coordonnées homogènes de l'épée par rapport au personnage : 
\[
H_p = 
\begin{pmatrix}
k.x & 0 & 0 \\
0 & k.y & 0 \\
0 & 0 & 1
\end{pmatrix}\]

D'après les exercices \ref{exo:1.1} et \ref{exo:1.2} : \newline
\[
T_e = 
\begin{pmatrix}
1 & 0 & 0,5 \\
0 & 1 & 0,5 \\
0 & 0 & 1
\end{pmatrix}
\quad \& \quad
R_e = 
\begin{pmatrix}
-1 & 0 & 0 \\
0 & -1 & 0 \\
0 & 0 & 1
\end{pmatrix}\]

On obtient par produit $T_e R_e$ la matrice de transformation locale de l'épée $TR_e$ par rapport au personnage : 
\[\boxed{
TR_e = 
\begin{pmatrix}
-1 & 0 & 0,5 \\
0 & -1 & 0,5 \\
0 & 0 & 1
\end{pmatrix}}\]

    \item Donner les coordonnées mondes de l'épée.\newline \newline
Réponse :

Pour obtenir ces coordonnées, on doit d'abord repèrer les coordonnées du personnage dans le monde. i.e sa matrice de transformation $TRH_p$ dans le monde qui est le produit $T_p R_p H_p$ :
\[\boxed{
TRH_p = 
\begin{pmatrix}
\sqrt{2} & -\sqrt{2} & 1 \\
\sqrt{2} & \sqrt{2} & 2 \\
0 & 0 & 1
\end{pmatrix}}\]

Puis par multiplication avec la matrice $TR_e$ on obtient la matrice C des coordonnées mondes de l'épée qui est :
\[\boxed{
C = 
\begin{pmatrix}
-\sqrt{2} & \sqrt{2} & 1+\sqrt{2} \\
-\sqrt{2} & -\sqrt{2} & 2 \\
0 & 0 & 1
\end{pmatrix}}\]

\end{enumerate}

\newpage

\sectionExercice{Implémentation des similitudes}
\label{exo:impl}
\begin{enumerate}
  \item Implémenter une classe vecteur et matrice adaptées pour les coordonnées homogènes.

Pour manipuler les coordonnées homogènes, nous avons implémenté une classe \texttt{Vector3} représentant un vecteur à trois
composantes, utilisé pour représenter les points en coordonnées homogènes. Cette classe fournit une méthode \texttt{multiply\_by\_matrix} permettant de transformer un vecteur par une matrice 3×3.

Nous avons ensuite défini une classe \texttt{Matrix3x3}, représentant une matrice générique en coordonnées homogènes. Elle inclut une méthode \texttt{prod} pour effectuer le produit matriciel. 

Ces nouvelles classes ont été ajoutées à notre bibliothèque Mathy dans \texttt{geometry\_engine\_librairie}.

  \item Implémenter les similitudes et des fonctions pour les manipuler.

Trois classes filles héritent de la classe \texttt{Matrix3x3} :
\begin{itemize}
\item \texttt{TranslationMatrix3x3}, qui crée une matrice de translation à partir de deux coordonnées $\begin{pmatrix} x & y \end{pmatrix}$,
\item \texttt{RotationMatrix3x3}, qui génère une matrice de rotation pour un angle donné en degrés,
\item \texttt{HomothetyMatrix3x3}, qui représente une homothétie (mise à l’échelle) selon un facteur donné.
\end{itemize}

Pour valider notre implémentation, nous avons repris l’énoncé de l’exercice 1 en Python. Chaque transformation (translation, rotation, homothétie) est d’abord instanciée séparément, puis combinée pour former une transformation globale du personnage. Une transformation locale est également définie pour l’épée, puis combinée avec celle du personnage. Enfin, pour obtenir les coordonnées mondes de l’épée, nous appliquons la transformation complète à l’origine du repère local de l’épée, représentée par le vecteur homogène $\begin{pmatrix} 0 & 0 & 1 \end{pmatrix}$.
\end{enumerate}

Le code de résolution de cet exercice est contenu dans le fichier \texttt{tp\_3\_exo1.py}, situé dans le dossier \texttt{geometry\_engine\_executables}.

Pour exécuter ce programme dans \textbf{Sublime Text}, presser \texttt{Ctrl + Shift + B}, puis sélectionner la commande \texttt{tp\_3\_exo1}.

\newpage

\sectionExercice{Orbite}

Soit $\mathcal{C}$ un cercle de rayon $r$ et de centre $M$ situé à une distance de $d$ de l'origine du monde. Soit $\mathcal{C}_1$ et $\mathcal{C}_2$ deux cercles de rayon $r_1$ et $r_2$ et placé à une distance $d_1$ et $d_2$ de $\mathcal{C}$. Ces deux cercles orbitent autour de $\mathcal{C}$ avec respectivement, une période de $t_1$ et $t_2$.
En utilisant les implémentations de l'exercice \ref{exo:impl}, simulez l'évolution de la trajectoire de ces trois cercles.


\end{document}
