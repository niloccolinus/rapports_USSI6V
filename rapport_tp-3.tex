% !TEX options=-synctex=-1
\documentclass[a4paper,12pt]{article}
\usepackage[utf8]{inputenc}
\usepackage[T1]{fontenc}
\usepackage[french]{babel}
\usepackage{amsmath,amssymb}
\usepackage{xcolor}
\usepackage{geometry}
\usepackage{soul}
\usepackage{graphicx} % pour inclure des images
\usepackage{animate}

\geometry{
  a4paper,
  total={170mm,257mm},
  left=20mm,
  top=20mm,
}

\usepackage{extensionCours}

\title{TP 3 - Similitudes}
\author{Groupe 2 \\* Colin, Henri, Leïla, Loïs}

\begin{document}
\maketitle

\sectionExercice{Théorie- Un bug dans Unity}
Vous êtes développeur d'un jeu sous Unity. Malheureusement, comme à son habitude le logiciel est défectueux et l'éditeur ne fonctionne plus du tout, de même que les fonctions qui permettent de définir la position et l'orientation manuellement. 
\begin{enumerate}
    \item \emph{Soit un personnage modélisé par un sprite instancié à l'origine du moteur, l'utilisateur souhaite le déplacer à la position $\begin{pmatrix} 1 & 2\end{pmatrix}$, donner la matrice de changement de coordonnée à appliquer au personnage pour lui donner sa nouvelle position.}\label{exo:1.1}\newline \newline
Réponse :

On cherche la matrice de transpositions correspondant aux coordonnées homogènes mondes du personnage et nommée $T_p$ : 

 \begin{equation}
T_p = 
\begin{pmatrix}
1 & 0 & a \\
0 & 1 & b \\
0 & 0 & 1
\end{pmatrix} \end{equation}

Tel que : 
 \begin{equation}\begin{pmatrix}
1 & 0 & a \\
0 & 1 & b \\
0 & 0 & 1
\end{pmatrix}
\begin{pmatrix}
0 \\
0 \\
1 
\end{pmatrix}
=
\begin{pmatrix}
1 \\
2 \\
1
\end{pmatrix} \end{equation}

Donc la matrice de transposition $T_p$ du personnage est égale à :
 \begin{equation}\boxed{
T_p = 
\begin{pmatrix}
1 & 0 & 1 \\
0 & 1 & 2 \\
0 & 0 & 1
\end{pmatrix}} \end{equation} \newline

    \item \emph{L'utilisateur souhaite également faire tourner le personnage de \(\frac{\pi}{4}\) dans le sens trigonométrique autour de l'origine. Donnez la matrice de rotation correspondante en coordonnées homogènes.} \label{exo:1.2}\newline \newline
Réponse :

On cherche la matrice de rotations correspondant aux coordonnées homogènes mondes du personnage et nommée $R_p$ : 
 \begin{equation}
R_p = 
\begin{pmatrix}
\cos(\theta) & -\sin(\theta) & 0 \\
\sin(\theta) & \cos(\theta) & 0 \\
0 & 0 & 1
\end{pmatrix} \end{equation}

Ici \(\theta\) = \(\frac{\pi}{4}\) donc : \newline
 \begin{equation}\boxed{
R_p = 
\begin{pmatrix}
\frac{\sqrt{2}}{2} & \frac{-\sqrt{2}}{2} & 0 \\
\frac{\sqrt{2}}{2} & \frac{\sqrt{2}}{2} & 0 \\
0 & 0 & 1
\end{pmatrix}} \end{equation} \newline
    \item \emph{On souhaite doubler la taille du personnage. Donnez la matrice d'homothétie correspondante.}\newline \newline
Réponse :

On cherche la matrice d'homotétie correspondant aux coordonnées homogènes mondes du personnage et nommée $H_p$ : 
 \begin{equation}
H_p = 
\begin{pmatrix}
k & 0 & 0 \\
0 & k & 0 \\
0 & 0 & 1
\end{pmatrix} \end{equation}

Ici $k$ = 2 donc : 
 \begin{equation}\boxed{
H_p = 
\begin{pmatrix}
2 & 0 & 0 \\
0 & 2 & 0 \\
0 & 0 & 1
\end{pmatrix}} \end{equation} \newline

    \item \emph{Considérez l'épée, qui doit être placée dans le repère local du personnage à la position $\begin{pmatrix} 0.5 & -0.5 \end{pmatrix}$. De plus l'épée est tournée de $\pi$ par rapport au personnage. Définissez la matrice de transformation locale à appliquer à l'épée par rapport au repère du personnage.}\newline \newline
Réponse :

On cherche les matrices de translation $T_e$ et de rotation $R_e$ correspondant aux coordonnées homogènes de l'épée par rapport au personnage : 
 \begin{equation}
H_p = 
\begin{pmatrix}
k.x & 0 & 0 \\
0 & k.y & 0 \\
0 & 0 & 1
\end{pmatrix} \end{equation}

D'après les exercices \ref{exo:1.1} et \ref{exo:1.2} : \newline
 \begin{equation}
T_e = 
\begin{pmatrix}
1 & 0 & 0,5 \\
0 & 1 & 0,5 \\
0 & 0 & 1
\end{pmatrix}
\quad \& \quad
R_e = 
\begin{pmatrix}
-1 & 0 & 0 \\
0 & -1 & 0 \\
0 & 0 & 1
\end{pmatrix} \end{equation}

On obtient par produit $T_e R_e$ la matrice de transformation locale de l'épée $TR_e$ par rapport au personnage : 
 \begin{equation}\boxed{
TR_e = 
\begin{pmatrix}
-1 & 0 & 0,5 \\
0 & -1 & 0,5 \\
0 & 0 & 1
\end{pmatrix}} \end{equation}

    \item \emph{Donner les coordonnées mondes de l'épée.}\newline \newline
Réponse :

Pour obtenir ces coordonnées, on doit d'abord repèrer les coordonnées du personnage dans le monde. i.e sa matrice de transformation $TRH_p$ dans le monde qui est le produit $T_p R_p H_p$ :
 \begin{equation}\boxed{
TRH_p = 
\begin{pmatrix}
\sqrt{2} & -\sqrt{2} & 1 \\
\sqrt{2} & \sqrt{2} & 2 \\
0 & 0 & 1
\end{pmatrix}} \end{equation}

Puis par multiplication avec la matrice $TR_e$ on obtient la matrice C des coordonnées mondes de l'épée qui est :
 \begin{equation}\boxed{
C = 
\begin{pmatrix}
-\sqrt{2} & \sqrt{2} & 1+\sqrt{2} \\
-\sqrt{2} & -\sqrt{2} & 2 \\
0 & 0 & 1
\end{pmatrix}} \end{equation}

Ainsi la position P de l'épée dans le monde est 
\[\boxed{
P =  (1+\sqrt{2};2)} 
\]

\end{enumerate}

\newpage

\sectionExercice{Implémentation des similitudes}
\label{exo:impl}
\begin{enumerate}
  \item \emph{Implémenter une classe vecteur et matrice adaptées pour les coordonnées homogènes.}

Pour manipuler les coordonnées homogènes, nous avons implémenté une classe \texttt{Vector3} représentant un vecteur à trois
composantes, et sa sous-classe \texttt{HomogeneousVector3} utilisée pour représenter les points en coordonnées homogènes. La classe \texttt{Vector3} fournit une méthode \texttt{multiply\_by\_matrix} permettant de transformer un vecteur par une matrice 3×3.

Nous avons ensuite défini une classe \texttt{Matrix3x3}, représentant une matrice générique en coordonnées homogènes. Elle inclut une méthode \texttt{prod} pour effectuer le produit matriciel. 

Ces nouvelles classes ont été ajoutées à notre bibliothèque Mathy dans \newline \texttt{geometry\_engine\_librairie}.

  \item \emph{Implémenter les similitudes et des fonctions pour les manipuler.}

Trois classes filles héritent de la classe \texttt{Matrix3x3} :
\begin{itemize}
\item \texttt{TranslationMatrix3x3}, qui crée une matrice de translation à partir de deux coordonnées $\begin{pmatrix} x & y \end{pmatrix}$,
\item \texttt{RotationMatrix3x3}, qui génère une matrice de rotation pour un angle donné en degrés,
\item \texttt{HomothetyMatrix3x3}, qui représente une homothétie (mise à l’échelle) selon un facteur donné.
\end{itemize}

Pour valider notre implémentation, nous avons repris l’énoncé de l’exercice 1 en Python. Chaque transformation (translation, rotation, homothétie) est d’abord instanciée séparément, puis combinée pour former une transformation globale du personnage. Une transformation locale est également définie pour l’épée, puis combinée avec celle du personnage. Enfin, pour obtenir les coordonnées mondes de l’épée, nous appliquons la transformation complète à l’origine du repère local de l’épée, représentée par le vecteur homogène $\begin{pmatrix} 0 & 0 & 1 \end{pmatrix}$.
\end{enumerate}

Le code de résolution de cet exercice est contenu dans le fichier \texttt{tp\_3\_exo1.py}, situé dans le dossier \texttt{geometry\_engine\_executables}.

Pour exécuter ce programme dans \textbf{Sublime Text}, presser \texttt{Ctrl + Shift + B}, puis sélectionner la commande \texttt{tp\_3\_exo1}.

\newpage

\sectionExercice{Orbite}
\setlength{\parskip}{0.5em}
\setlength{\parindent}{0em}

Soit $\mathcal{C}$ un cercle de rayon $r$ et de centre $M$ situé à une distance $d$ de l'origine du monde. Soit $\mathcal{C}_1$ et $\mathcal{C}_2$ deux cercles de rayon $r_1$ et $r_2$ et placés à une distance $d_1$ et $d_2$ de $\mathcal{C}$. Ces deux cercles orbitent autour de $\mathcal{C}$ avec respectivement, une période de $t_1$ et $t_2$.

\vspace{0.5cm}

\emph{En utilisant les implémentations de l'exercice précédent, simulez l'évolution de la trajectoire de ces trois cercles.}

\begin{figure}[h]
    \centering
    \animategraphics[autoplay, loop, width=200px]{30}{illustrations/orbite-}{0}{79}
    \caption{Simulation de l'évolution des trajectoires des cercles $\mathcal{C}_1$, $\mathcal{C}_2$ et $\mathcal{C}$.}
    \label{fig:orbite}
\end{figure}

En se basant sur les classes implémentées dans l'exercice précédent et le moteur de rendu de la librairie pygame, 
nous avons créé une simulation animée de l'évolution en temps réel des trajectoires des cercles $\mathcal{C}_1$, $\mathcal{C}_2$ et $\mathcal{C}$.

Les valeurs numériques suivantes sont définies pour configurer la simulation :

\begin{itemize}
    \item Un rayon ($r$, $r_1$, $r_2$) pour chaque cercle,
    \item Une distance ($D$, $D_1$, $D_2$) qui représente la distance entre le centre de chaque cercle et l'origine du monde (pour $\mathcal{C}$) ou entre le centre de $\mathcal{C}$ et les cercles $\mathcal{C}_1$ et $\mathcal{C}_2$,
    \item Une vélocité ($\omega$, $\omega_1$, $\omega_2$) pour chaque cercle, qui détermine la vitesse de rotation, calculée par la formule $\omega = \frac{360}{T}$, où $T$ est la période de rotation exprimée en secondes,
    \item Une période ($T$, $T_1$, $T_2$) pour chacun des cercles, qui détermine la durée d'une rotation complète,
    \item Un angle de rotation ($\theta$, $\theta_1$, $\theta_2$), exprimé en degrés, pour chaque cercle, qui est calculé à chaque itération de la simulation en fonction de la vélocité et du temps écoulé $dt$.
\end{itemize}

Le cercle $\mathcal{C}$ tourne autour de l'origine, tandis que les cercles $\mathcal{C}_1$ et $\mathcal{C}_2$ tournent en orbite autour de $\mathcal{C}$.

La position de chaque cercle est mise à jour à chaque itération de la simulation (chaque frame) en appliquant la matrice de rotation correspondante à l'angle de rotation calculé.

Cette rotation est effectuée par la méthode \texttt{rotate-point} du script de l'exercice. 
Elle applique à un point donné une rotation par rapport à un angle et une origine spécifiés. La méthode procède ainsi :

\begin{itemize}
    \item Elle applique une translation pour déplacer le point donné à l'origine du repère.
    \item Elle applique la matrice de rotation à ce point pour obtenir la nouvelle position du point dans le repère local.
    \item Elle applique une translation inverse pour ramener le point à sa position d'origine.
\end{itemize}

Les méthodes \texttt{RotationMatrix3x3} et \texttt{TranslationMatrix3x3} sont utilisées pour créer les matrices de rotation et de translation respectivement. 
Ces matrices sont ensuite multipliées par le vecteur homogène représentant la position du point donné.

L'homogénéité des coordonnées permet de manipuler les points en 2D dans un espace 3D, par une troisième coordonnée égale à 1.

Le module \texttt{Renderer}, qui utilise la librairie pygame, est responsable de l'affichage des cercles et de la mise à jour de leur position, en redessinant chaque cercle à leur nouvelle position à chaque itération.

Le code de résolution de cet exercice est contenu dans le fichier \texttt{tp\_3\_exo3.py}, situé dans le dossier \texttt{geometry\_engine\_executables}.

Pour exécuter ce programme dans \textbf{Sublime Text}, presser \texttt{Ctrl + Shift + B}, puis sélectionner la commande \texttt{tp\_3\_exo3}.

\end{document}
